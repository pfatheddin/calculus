\documentclass{ximera}

\input{../../preamble.tex}

\author{Jason Miller}
\license{Creative Commons 3.0 By-NC}


\outcome{}


\begin{document}
\begin{exercise}
Determine the integral.
\[
\int x\arctan(x) \d x
\]


The integrand is a product where one factor is easy to integrate and the other factor is easy to differentiate. Since nothing else seems to be helpful, we try integration by parts. 

Let $u=\answer{ \arctan(x)}$ and $\d v= \answer{  x } \d x$. 

Then 
\[
  \d u= \answer{ \frac{1}{1+x^{2}}} \text{ and }  v=\answer{ \frac{x^{2}}{2} }
\]

\begin{exercise}

Using integration by parts, our original integral becomes

\[
\int x\arctan(x) \d x=\answer{\frac{x^{2}\arctan(x)}{2}} - \int \answer{\frac{x^{2}}{2(1+x^{2})}} \d x
\]

\begin{exercise}

The second integral $\int \frac{x^{2}}{2(1+x^{2})} \d x$ can be done with a trig substitution. 

Let $x=\answer{ \tan(\theta)}$. Then $\d x=\answer{ \sec^{2}(\theta)} \d \theta$.

The integral in terms of $\theta$ is

\[
\int \frac{x^{2}}{2(1+x^{2})} \d x =\int \answer{ \frac{\tan^{2}(\theta)}{2}} \d \theta
\]

\begin{exercise}
The antiderivative in terms of $\theta$ is

\[
\int \frac{\tan^{2}(\theta)}{2} \d \theta=\answer{\frac{1}{2}\left( \tan(\theta) - \theta \right) + C }
\]
(Use $C$ for the constant of integration and recall the identity $tan^2(x)+1=sec^2(x)$)

\begin{exercise}
Going back to our original variable $x$ we have 

\[
\int \frac{x^{2}}{2(1+x^{2})} \d x=\answer{\frac{1}{2}\left( x-\arctan(x) \right) + C }
\]
(Use $C$ for the constant of integration)

\begin{exercise}

Now we can combine this with the result we obtained from integration by parts to get


\[
\int x\arctan(x) \d x=\answer{ \frac{x^{2}\arctan(x)}{2}-\frac{x}{2}+\frac{\arctan(x)}{2} + C}
\]
(Use $C$ for the constant of integration)



\end{exercise}
\end{exercise}

\end{exercise}
\end{exercise}
\end{exercise}
\end{exercise}
\end{document}
