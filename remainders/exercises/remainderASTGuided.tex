\documentclass{ximera}

\input{../../preamble.tex}

\author{Jim Talamo}
\license{Creative Commons 3.0 By-NC}


\outcome{Understand the alternating series test estimates.}

\begin{document}

\begin{exercise}
Consider the series $\sum_{k=0}^{\infty} \frac{(-1)^k}{k!}$.  

Does the alternating series test apply?

\begin{multipleChoice}
\choice[correct]{Yes.}
\choice{No.}
\end{multipleChoice}

What does the alternating series test tell us?
\begin{multipleChoice}
\choice{The series converges to $0$.}
\choice[correct]{The series converges, but we do not know its value yet.}
\choice{The series diverges.}
\end{multipleChoice}

\begin{feedback}
Note that the series is of the form $\sum_{k=0}^{\infty} (-1)^k a_k$, where $a_k = \frac{1}{k!}>0$.  Since $\lim_{n \to \infty} \frac{1}{n!} =0$ and $a_{n+1} \leq a_n$ for all $n$, we conclude that the series converges by the alternating series test.  
\end{feedback}

\begin{exercise}
Recall that once we have determined that a series converges, we are often interested in the following two questions.

\begin{itemize}
\item[1.] How bad is the error made when we approximate a convergent infinite series by its first several terms?
%In other words, if we specify $N$, how close is $\sum_{k=n_0}^{N} a_k$ to the exact value of $\sum_{k=n_0}^{\infty} a_k$?
\item[2.] How many terms should we specify if we want to know the value of a convergent series to obtain a desired precision?
%Said another way, given an acceptable value for the error, what value should we pick for $N$ so $\sum_{k=n_0}^{N} a_k$ approximates $\sum_{k=n_0}^{\infty} a_k$ that accurately?
\end{itemize}

We now want to approximate the value of $\sum_{k=0}^{\infty} \frac{(-1)^k}{k!}$.  Since we do not have a way to find an explicit formula for $s_n=\sum_{k=1}^n \frac{(-1)^k}{k!}$, we need to turn to the remainder estimate that comes with alternating series.

\begin{theorem}[Alternating Series Remainder Estimates]
If $\{a_n\}_{n=n_0}$ be a sequence whose terms are positive and decreasing and
$\lim_{n\to\infty} a_n=0$. Then,  
\[
\big| r_n \big| \leq a_{n+1} \qquad \textrm{ for all } n \geq n_0,
\]
where $r_n = \sum_{k=n+1}^{\infty} a_k$.
\end{theorem}

Unlike the integral test, we will use $s_n$ as our approximation to $\sum_{k=0}^{\infty} \frac{(-1)^k}{k!}$ since the sequence $\{s_n\}_{n=1}$ is \wordChoice{\choice{eventually}\choice[correct]{never eventually}} monotonic.  This is also why we consider the \emph{magnitude} of the error; whether $s_n$ is an overestimate or underestimate will depend on the choice of $n$. 

%%%%%%%%%%%%%%%%%%%%%%%%%%%%%%%%%%%%%%%%%%%%%
\begin{exercise}
Calculate $\sum_{k=0}^{4} \frac{(-1)^k}{k!}$ to four decimal places.  

To four decimal places, we find that $\sum_{k=0}^{4} \frac{(-1)^k}{k!} = \answer[tolerance=.0001]{.3750}$.  

We use the alternating series results to find the maximum possible error made if $\sum_{k=0}^{4} \frac{(-1)^k}{k!}$ is used to approximate $\sum_{k=0}^{\infty} \frac{(-1)^k}{k!}$. 

First, note that for our series, $a_n =$ \wordChoice{\choice{$\frac{(-1)^n}{n!}$}\choice[correct]{$\frac{1}{n!}$}}.  Since we are using $\sum_{k=0}^{4} \frac{(-1)^k}{k!}$ for our approximation, we should choose $n=\answer{4}$.  What do the remainder results tell us that the magnitude of the error?

\begin{multipleChoice}
\choice{$\big|r_4 \big| \leq \frac{1}{3!}$}
\choice{$\big|r_4 \big| \leq \frac{1}{4!}$}
\choice[correct]{$\big|r_4 \big| \leq \frac{1}{5!}$}
\end{multipleChoice}

To four decimal places, what is the maximum possible error made if we use $\sum_{k=0}^{4} \frac{(-1)^k}{k!}$ as our approximation?

The maximum possible error made is $\answer[tolerance=.0001]{.0083}$.

\begin{feedback}
We thus conclude that $\sum_{k=0}^{\infty} \frac{(-1)^k}{k!} \approx .3750$ and that this approximation is accurate to within $.0083$.
\end{feedback}

%%%%%%%%%%%%%%%%%%%%%%%%%%%%%%%%%%%%%%%%%%%%%
\begin{exercise}
Suppose that we want to approximate $\sum_{k=0}^{\infty} \frac{(-1)^k}{k!}$ to within $.00001$ of its actual value.  We will do so by finding a value of $N$ for which we are guaranteed that $\big|r_N\big| \leq .00001$, then use $s_N$ to provide the approximation.

We do not have a formula for $r_n$, so we must turn to the remainder results.  Note that we have $\big|r_n\big| \leq a_{n+1}$, so if we ensure that $a_{n+1} \leq .00001$, we know that the magnitude of the error will be no more than $.00001$.

\[
\big|r_n\big| \leq \frac{1}{(n+1)!} \leq .00001
\]

There isn't a nice general method for solving equations with factorials, but we can check if the above inequality holds for various $n$.

\begin{itemize}
\item If we choose $n=5$, then to seven decimal places $\frac{1}{(n+1)!} = \frac{1}{6!} \approx \answer{.0013889}$, so $n=5$ is \wordChoice{\choice[correct]{too small}\choice{an acceptable value for $n$}} .
\item If we choose $n=7$, then to seven decimal places $\frac{1}{(n+1)!} = \frac{1}{8!} \approx \answer{.0000248}$, so $n=7$ is \wordChoice{\choice[correct]{too small}\choice{an acceptable value for $n$}} .
\item If we choose $n=8$, then to seven decimal places $\frac{1}{(n+1)!} = \frac{1}{9!} \approx .000000276$, so $n=8$ will work.
\end{itemize}

We thus compute $\sum_{k=0}^8 \frac{(-1)^k}{k!} \approx \answer[tolerance=.000003]{.367882}$ to six decimal places and note that this will approximate $\sum_{k=0}^\infty \frac{(-1)^k}{k!}$ to within $.00001$ of its true value.
\end{exercise}

\end{exercise}
%%%%%%%%%%%%%%%%%%%%%%%%%%%%%%%%%%%%%%%%%%%%%

\end{exercise}
\end{exercise}
\end{document}
