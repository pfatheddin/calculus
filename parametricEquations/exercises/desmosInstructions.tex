\documentclass{ximera}

\input{../../preamble.tex}

%\outcome{Find tangent lines to parametric curves}
\author{Jim Talamo}

\begin{document}
\begin{exercise}
Desmos is a very useful tool and can be used to develop good intuition about parametric equations.  The steps below show you how to use Desmos to animate a parametrically defined curve in the context of a specific example.

Suppose that a curve $C$ is defined parametrically by:

\[
\begin{cases}
x(t)&=  t^4-2t^3+5t-4\\
y(t)&=  3t^3-4t^2-5t+3
\end{cases} 
, -3 < t < 3
\]

To see this curve:

\begin{itemize}
\item[1.] Go to https://www.desmos.com/ and click ``Start Graphing".
\item[2.] Type the following expressions \emph{exactly} as written below:
\begin{itemize}
\item In Line 1, type: \verb|X(t) = t^4-2t^3+5t-4|  
\item In Line 2, type: \verb|Y(t) = 3t^3-4t^2-5t+3|
\item Then, uncheck the boxes left of the expressions to make the graphs disappear.
\item In Line 3, type \verb|(X(t),Y(t))| and select $-3\leq t \leq3$ when the bounds for $t$ arise.  Check on the box next to what you typed, and you should see an interesting curve.
\end{itemize}
Can this curve be represented by a single function?

\begin{multipleChoice}
\choice{Yes; any vertical line will intersect the curve in no more than one place.}
\choice[correct]{No; a vertical line will intersect the curve in more than one place.}
\end{multipleChoice}

\item[3.] Now, uncheck the box next to: \verb|(X(t),Y(t))|
\item[4.] In Line 4, type \verb|(X(a),Y(a))|.  When ``add slider'' pops up as an option, click on it, and choose the same range of values for $a$ as you did for $t$.  That is, select $-3 \leq a \leq 3$.
\item[5.] Go back to Line 3 and add ``$\{t<a\}$" right after $Y(t)$ (before the final parenthesis).  

Line 3 should now read $(X(t),Y(t)\{t<a\} )$.

\item[6.] Now click on the Play button next to $a$.  You should see an animation for the curve.  
\begin{itemize}
\item Experiment with the speed of the animation by by clicking on the expression in the ``$<< \quad >>$'' box (which appears only while the curve is being animated)
\item Experiment with the direction in which the curve is traced out by clicking on the arrows that appear to the left of the ``$<< \quad >>$'' box.  When both arrow point right, this shows the \emph{positive orientation} (the direction in which the curve is traversed as $t$ increases).
\end{itemize}
\end{itemize}
Now answer the following questions about the curve:

For the $t$-value where $x=6$, we find $y=\answer{1}$.

The curve intersects the $x$-axis $\answer{3}$ times between $-6 \leq x \leq 6$.

\begin{remark}
When you are working problems from these exercises, you should check your work using Desmos!
\end{remark}

\end{exercise}
\end{document}
