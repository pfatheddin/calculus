\documentclass{ximera}

\input{../../preamble.tex}

\outcome{Recall the meaning and properties of inverse trigonometric functions.}

\author{Nela Lakos \and Kyle Parsons}

\begin{document}
\begin{exercise}

The exact value of $\sin(\tan^{-1}(5))$ is
\[
\sin(\arctan(5)) = \answer{\frac{5}{\sqrt{26}}}.
\]

Find the derivative.
\[
\ddx\left[\vphantom{\ddx}\sin(\arctan(x))\right] = \answer{\cos(\arctan(x))\frac{1}{1+x^2}}
\]

Assuming $x>0$ we can use a right triangle to simplify $\sin(\tan^{-1}(x))$ as
\[
\sin(\arctan(x)) = \answer{\frac{x}{\sqrt{1+x^2}}}.
\]

Using the above expression to take the derivative we have
\[
\ddx\left[\vphantom{\ddx}\sin(\arctan(x))\right] = \answer{\frac{1}{(1+x^2)^{\frac{3}{2}}}}.
\]

In order to see that the two expressions for the derivative of $\sin(\arctan(x))$ are the same, simplify $\cos(\arctan(x))$ as
\[
\cos(\arctan(x)) = \answer{\frac{1}{\sqrt{1+x^2}}}.
\]

\end{exercise}
\end{document}
